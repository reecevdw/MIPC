% Summarize Finland’s biodiversity-finance challenge and define BWF’s reoriented role.

% Include:
% 	•	Finland’s forests → carbon sink turned source (2021).
% 	•	Dual mandate tension: biodiversity restoration vs. financial sustainability.
% 	•	Your strategic framing:
% “BWF must act as Finland’s Natural Capital Fund—converting ecological resilience into investable capital.”

% Deliverable: Define your investment philosophy (why biodiversity = value).
% E.g., “Biodiversity enhances asset durability and mitigates regulatory and stranded-asset risk—creating unpriced alpha.”

\section*{Mandate Reframing}

Finland’s forests, which cover over 75\% of the country’s land area, have long served as both an ecological foundation and an economic engine. However, in 2021 these forests shifted from being a net carbon sink to a net carbon source, driven by intensive harvesting and peatland degradation. This reversal jeopardizes Finland’s legally binding 2035 net-zero emissions target and threatens the economic stability of a sector responsible for nearly 4\% of national GDP and 18\% of exports. 

The Borealis Wealth Fund (BWF), with €10 billion in assets under management and a recurring inflow of approximately €1 billion per year from forestry-linked tax revenues, has been tasked with realigning the country’s natural capital strategy. Initially launched as a broad green investment vehicle, BWF must now address a dual mandate: to generate long-term, risk-adjusted returns while restoring the ecological resilience that underpins Finland’s bioeconomy. This represents not merely a portfolio adjustment, but a systemic reframing of how national wealth and nature interact.

We propose repositioning BWF as Finland’s \textbf{Natural Capital Sovereign Fund}—a platform that treats biodiversity as a core economic asset. Under this model, BWF channels forestry-derived revenues into investments that enhance ecosystem health, stabilize carbon sinks, and de-risk Finland’s export base from future EU biodiversity and land-use regulations. The Fund’s objective is to deliver sustainable financial returns (targeting 5–6\% per annum) while achieving measurable biodiversity outcomes, including the restoration of over 200,000 hectares of degraded forest and peatland by 2035.

This reframing rests on a clear investment philosophy:

\begin{quote}
\emph{Biodiversity enhances asset durability and long-term value creation by improving ecosystem productivity, regulatory resilience, and social license to operate. Protecting nature is not a constraint on returns—it is a catalyst for enduring value.}
\end{quote}

BWF’s redefined investment approach will be guided by three principles:
\begin{enumerate}[label=\textbf{\arabic*.}]
    \item \textbf{Integration:} Embed biodiversity risk and opportunity into every investment decision using frameworks such as the Taskforce on Nature-related Financial Disclosures (TNFD) and the EU Biodiversity Strategy for 2030.
    \item \textbf{Additionality:} Allocate at least 30\% of annual new capital inflows to projects that deliver measurable ecological benefits—such as habitat restoration, peatland recovery, and native species protection—that exceed baseline market activity.
    \item \textbf{Resilience:} Diversify across asset classes and geographies to reduce overexposure to Finland’s forestry cycle, ensuring long-term capital preservation and compliance with evolving EU nature restoration laws.
\end{enumerate}

Through this mandate, BWF evolves from a conventional green finance fund into a global model for biodiversity-aligned sovereign investment. By quantifying and monetizing the value of natural capital, BWF will enable Finland to transform ecological preservation into a sustainable source of fiscal strength and international leadership.