\section*{Diagnostic: Legacy Portfolio \& Gaps}

\subsection*{Where We Are}
Borealis Wealth Fund (BWF) manages \textbf{€10B} and has delivered an average \textbf{6.5\%} annual return since launch. The legacy portfolio is diversified across public markets, fixed income, and real assets, but ESG screening has not explicitly targeted biodiversity outcomes.

\begin{table}[h!]
\centering
\resizebox{0.9\linewidth}{!}{%
\begin{tabular}{lccc}
\toprule
\textbf{Asset Class} & \textbf{Share of AUM (\%)} & \textbf{Role Today} & \textbf{Biodiversity Link} \\
\midrule
Public Equities (ESG-screened) & \textasciitilde 40 & Growth \& liquidity & Indirect / weak \\
Green Bonds \& Thematic Credit & \textasciitilde 15 & Income \& duration & Mixed / label-driven \\
Green Infrastructure (wind/solar/transit) & \textasciitilde 20 & Inflation hedge, stable cash & Indirect (climate-first) \\
Forestry \& Land Holdings & \textasciitilde 15 & Real-asset beta & Direct, but unmanaged \\
Cash \& Liquidity & \textasciitilde 10 & Dry powder & None \\
\bottomrule
\end{tabular}%
}
\caption{Legacy portfolio overview by strategic sleeve (percent shares inferred from case totals).}
\end{table}

\noindent \textbf{Geography:} \textasciitilde 55\% Finland, 30\% EU ex-Finland, 10\% North America, 5\% EM.

\vspace{0.5em}
\subsection*{What Works}
\begin{itemize}[leftmargin=*, itemsep=2pt]
    \item \textbf{Solid core beta:} Low-cost public equity exposure and green infrastructure provide diversified return drivers and partial inflation protection.
    \item \textbf{Policy alignment (climate):} Green infrastructure and thematic credit align with climate transition risks and EU taxonomy to a degree.
    \item \textbf{Liquidity balance:} A meaningful liquid sleeve (equities, credit, cash) supports rebalancing and new program launches.
\end{itemize}

\subsection*{Binding Constraints (Context)}
\begin{itemize}[leftmargin=*, itemsep=2pt]
    \item \textbf{2035 net-zero pathway:} Finland’s forests flipped from a carbon sink to a source (2021), compressing the national carbon budget and elevating land-use scrutiny.
    \item \textbf{EU nature regulation:} The EU Biodiversity Strategy \& Nature Restoration Law increase the risk of stranded or down-classified forestry assets without measurable biodiversity outcomes.
    \item \textbf{Trade \& competitiveness:} Export partners increasingly demand traceable, biodiversity-positive supply chains; volume-only growth is fragile.
\end{itemize}

\subsection*{Identified Gaps}
\begin{enumerate}[leftmargin=*, itemsep=2pt]
    \item \textbf{ESG $\neq$ Biodiversity:} Current ESG screens do not capture habitat integrity, species abundance, deadwood density, or landscape connectivity. Result: \emph{impact dilution}.
    \item \textbf{Unpriced Nature Risk:} Valuation models omit biodiversity loss channels (soil health, pest risk, water regulation) and \emph{regulatory repricing}. Result: \emph{beta drag \& stranded-asset risk}.
    \item \textbf{Concentration Risk (Domestic Forestry):} Large home bias to Finland’s forestry cycle without explicit biodiversity management. Result: \emph{correlated downside under policy or ecological shocks}.
    \item \textbf{Impact Additionality \& Traceability:} The portfolio lacks auditable, decision-useful nature KPIs (e.g., hectares restored, species richness, peatland water-table recovery). Result: \emph{weak accountability}.
    \item \textbf{Liquidity Mismatch:} Real-asset sleeves (forestry, infrastructure) are illiquid relative to mandate agility; no dedicated \emph{impact working capital} to fund restoration ramps. Result: \emph{slow pivot}.
    \item \textbf{Revenue–Nature Feedback Loop Missing:} Forestry-linked inflows are not programmatically recycled into biodiversity restoration at scale. Result: \emph{mission drift}.
\end{enumerate}

\subsection*{Risk Map (Now)}
\begin{table}[h!]
\centering
\begin{tabular}{p{3.3cm} p{6.6cm} p{3.3cm}}
\toprule
\textbf{Risk} & \textbf{Mechanism} & \textbf{Current Exposure} \\
\midrule
Regulatory / Policy & Tightening EU nature rules; taxonomy eligibility; disclosure mandates & High (forestry, land) \\
Physical / Ecological & Carbon sink reversal; species decline; soil/water degradation & Medium–High (domestic) \\
Market / Reputation & Buyer standards for biodiversity; risk of greenwashing claims & Medium (public equities, bonds) \\
Liquidity / Funding & Illiquid real assets vs. need for rapid reallocation to restoration & Medium \\
Concentration / Home Bias & Finland forestry cycle, export cyclicality & High \\
\bottomrule
\end{tabular}
\caption{Principal risk channels and qualitative exposure assessment.}
\end{table}

\subsection*{Root Cause Synthesis}
\begin{itemize}[leftmargin=*, itemsep=2pt]
    \item Portfolio construction optimized for \emph{climate mitigation labels}, not \emph{biodiversity outcomes}.
    \item Absence of a \textbf{biodiversity-adjusted return} lens limits capital rotation toward nature-positive assets.
    \item Governance and data pipelines do not yet support \emph{TNFD-aligned} measurement or auditability.
\end{itemize}

\subsection*{Strategic Need (So What)}
\begin{quote}
\textbf{Shift from green finance to natural capital finance.} Embed biodiversity as a priced risk factor and investable outcome, rebalancing toward assets with measurable ecological restoration while preserving the Fund’s long-run 5–6\% return target.
\end{quote}

\subsection*{Implications for Strategy (Bridge to Next Section)}
\begin{itemize}[leftmargin=*, itemsep=2pt]
    \item Establish a \emph{Biodiversity Investment Loop} (Screen $\rightarrow$ Select $\rightarrow$ Transform $\rightarrow$ Report) with TNFD-aligned KPIs.
    \item Introduce a \textbf{Biodiversity-Adjusted Return (BAR)} framework to evaluate all assets.
    \item Carve out liquid \emph{transition capital} to accelerate peatland/old-growth restoration and biodiversity-linked instruments.
    \item Reduce home-bias risk via diversified natural capital allocations (geography, instruments), maintaining mission-critical domestic programs.
\end{itemize}

