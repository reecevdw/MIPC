\section*{Diagnostic: Legacy Portfolio \& Gaps}

\subsection*{Where We Are}
Borealis Wealth Fund (BWF) manages \textbf{€10B} and has delivered an average \textbf{6.5\%} annual return since launch. The legacy portfolio is diversified across public markets, fixed income, and real assets, but ESG screening has not explicitly targeted biodiversity outcomes.

\begin{table}[h!]
\centering
\resizebox{0.9\linewidth}{!}{%
\begin{tabular}{lccc}
\toprule
\textbf{Asset Class} & \textbf{Share of AUM (\%)} & \textbf{Role Today} & \textbf{Biodiversity Link} \\
\midrule
Public Equities (ESG-screened) & \textasciitilde 40 & Growth \& liquidity & Indirect / weak \\
Green Bonds \& Thematic Credit & \textasciitilde 15 & Income \& duration & Mixed / label-driven \\
Green Infrastructure (wind/solar/transit) & \textasciitilde 20 & Inflation hedge, stable cash & Indirect (climate-first) \\
Forestry \& Land Holdings & \textasciitilde 15 & Real-asset beta & Direct, but unmanaged \\
Cash \& Liquidity & \textasciitilde 10 & Dry powder & None \\
\bottomrule
\end{tabular}%
}
\caption{Legacy portfolio overview by strategic sleeve (percent shares inferred from case totals).}
\end{table}

\noindent \textbf{Geography:} \textasciitilde 55\% Finland, 30\% EU ex-Finland, 10\% North America, 5\% EM.

\vspace{0.5em}
\subsection*{What Works}
\begin{itemize}[leftmargin=*, itemsep=2pt]
    \item \textbf{Solid core beta:} Low-cost public equity exposure and green infrastructure provide diversified return drivers and partial inflation protection.
    \item \textbf{Policy alignment (climate):} Green infrastructure and thematic credit align with climate transition risks and EU taxonomy to a degree.
    \item \textbf{Liquidity balance:} A meaningful liquid sleeve (equities, credit, cash) supports rebalancing and new program launches.
\end{itemize}
\subsection*{Binding Constraints }
\begin{itemize}[leftmargin=*, itemsep=2pt]
    \item \textbf{2035 Net-Zero Pathway:} Finland’s forests shifted from a carbon sink to a carbon source in 2021, compressing the national carbon budget and intensifying scrutiny on land-use and forestry practices.
    \item \textbf{EU Nature Regulation:} The EU Biodiversity Strategy and Nature Restoration Law increase compliance pressure, making traditional high-yield forestry less viable without demonstrable ecological outcomes.
    \item \textbf{Trade and Competitiveness:} International buyers, especially within the EU, increasingly demand traceable, biodiversity-positive supply chains. Failure to meet these standards risks market access and export competitiveness.
\end{itemize}

\subsection*{Identified Gaps}
\begin{enumerate}[leftmargin=*, itemsep=2pt]
    \item \textbf{ESG $\neq$ Biodiversity:} Current ESG screens do not capture habitat integrity, species abundance, deadwood density, or landscape connectivity. 
    \item \textbf{Logging Yields vs. Future EU Regulations:} EU biodiversity laws and restoration targets will likely constrain harvest volumes. The Fund’s return model must anticipate this by investing in higher-value wood products and wood-based innovation, enabling profitability even with lower yields.
    \item \textbf{Liquidity vs. Long-term Natural Assets:} Real assets such as forestland and green infrastructure generate durable impact but remain illiquid. Without sufficient liquidity, the Fund risks being unable to pivot or scale new biodiversity projects. A structured cash and green bond allocation can serve as \emph{impact working capital} to fund near-term transitions.
    \item \textbf{Domestic Exposure vs. Global Trade Partners:} While increasing Finland’s forestry investment supports domestic biodiversity goals, overconcentration exposes the Fund to local regulatory, ecological, and market shocks. International ESG equities and green infrastructure holdings mitigate this by sustaining diversified trade and financial linkages.
\end{enumerate}

\subsection*{Risk Map (Now)}
\begin{table}[h!]
\centering
\begin{tabular}{p{3.3cm} p{6.6cm} p{3.3cm}}
\toprule
\textbf{Risk} & \textbf{Mechanism} & \textbf{Current Exposure} \\
\midrule
Regulatory / Policy & Tightening EU nature and restoration mandates impacting harvest volumes & High (forestry, land) \\
Physical / Ecological & Carbon sink reversal; biodiversity loss; soil and water stress & Medium–High (domestic forests) \\
Market / Reputation & Global buyers’ biodiversity standards; reputational risk from greenwashing & Medium (equities, bonds) \\
Liquidity / Funding & High share of illiquid real assets vs. short-term capital needs & Medium \\
Concentration / Home Bias & Exposure to Finland’s forestry and export cycles & High \\
\bottomrule
\end{tabular}
\caption{Principal risk channels and qualitative exposure assessment.}
\end{table}


\subsection*{Implications for Strategy }
\begin{itemize}[leftmargin=*, itemsep=2pt]
    \item Implement a strategy that focuses on net-zero goals while also addressing biodiversity through maintaining forested lands investments and reforesting new land.
    \item Establish a \emph{Value-over-Volume} logging model, reinvesting profits into R\&D and venture capital that increase wood product margins.
    \item Maintain 15\%–20\% of the portfolio in liquid green bonds and cash to ensure flexibility in meeting restoration and reforestation funding needs.
    \item Strengthen cross-border ESG equity exposure to sustain Finland’s trade relationships and hedge domestic forestry concentration risk.
    
\end{itemize}
