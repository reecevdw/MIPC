% Purpose: Hook the judges immediately; make your dual mandate solution and fund vision crystal clear.

% Suggested structure:
% 	•	One-sentence purpose:
% “We propose a redesigned Borealis Wealth Fund that converts Finland’s forestry revenues into a self-sustaining biodiversity investment engine, balancing 5–6% annual financial returns with measurable biodiversity restoration by 2035.”
% 	•	3 pillars of your solution (e.g. “Biodiversity Integration, Portfolio Transformation, and Global Co-Leadership”).
% 	•	Key metrics: AUM (€10B), expected return, biodiversity targets (hectares restored, carbon sink recovery timeline).
% 	•	Snapshot of your portfolio allocation (e.g., pie chart).
% 	•	Signature innovation (e.g. “Natural Capital Performance Index” or “Biodiversity-Linked Bonds”).

\section*{Executive Summary:}

\par The BWF Asset Management Strategy was designed to deliver long-term returns while supporting Finland’s carbon-neutral ambitions. However, the nation’s timber industry—central to its economy—often conflicts with these goals, and ongoing deforestation has accelerated biodiversity loss. Addressing these challenges requires a balanced approach that integrates environmental restoration with economic sustainability.
\par The revised strategy seeks to achieve carbon neutrality through biodiversity, not despite it. BWF will invest in native forest restoration and reforestation projects that enhance long-term carbon capture and ecosystem health, generating returns through land value appreciation and future sustainable timber revenues.
\par To balance environmental and economic priorities, the fund will back companies and innovations that raise the value per unit of wood, enabling reduced harvest volumes without sacrificing profitability. While increasing domestic exposure to Finland’s forests, BWF will maintain diversification through international equities and green infrastructure to manage risk.
\par The fund will also scale biodiversity-focused projects with lower immediate returns, offsetting them through strong performance in ESG-aligned equities, and will promote sustainable trade partnerships, advocating EU preferences for conservation-certified Finnish timber.
\par The \$10 billion portfolio allocates 50\% to global ESG equities, 20\% to forestry and land holdings, 15\% to green infrastructure, 10\% to green bonds, and 5\% to cash.
This balanced strategy positions BWF as a global model for nature-positive finance, demonstrating that economic prosperity, carbon neutrality, and biodiversity can reinforce each other.
