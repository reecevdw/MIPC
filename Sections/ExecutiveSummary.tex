% Purpose: Hook the judges immediately; make your dual mandate solution and fund vision crystal clear.

% Suggested structure:
% 	•	One-sentence purpose:
% “We propose a redesigned Borealis Wealth Fund that converts Finland’s forestry revenues into a self-sustaining biodiversity investment engine, balancing 5–6% annual financial returns with measurable biodiversity restoration by 2035.”
% 	•	3 pillars of your solution (e.g. “Biodiversity Integration, Portfolio Transformation, and Global Co-Leadership”).
% 	•	Key metrics: AUM (€10B), expected return, biodiversity targets (hectares restored, carbon sink recovery timeline).
% 	•	Snapshot of your portfolio allocation (e.g., pie chart).
% 	•	Signature innovation (e.g. “Natural Capital Performance Index” or “Biodiversity-Linked Bonds”).

\documentclass[12pt]{article}
\usepackage{listings} % Include the listings package
\usepackage[T1]{fontenc} % Add the fontenc package with T1 encoding
\usepackage{subcaption}
\usepackage{tikz}
\usepackage{stmaryrd}

\usetikzlibrary{automata, positioning, arrows}

\usepackage{./Layout/EngReport}

\graphicspath{{Images/}}
\bibliography{./Layout/Sources}
\onehalfspacing
\graphicspath{Images/}
\geometry{letterpaper, portrait, includeheadfoot=true, hmargin=1in, vmargin=1in}

\usepackage{amsfonts,amsmath,amssymb,amsthm}
\usepackage{multicol}
\usepackage{microtype}
\usepackage{soul}
\usepackage{tikz}
\usepackage{parskip}
\usepackage{booktabs}
\usepackage{enumitem}
\usetikzlibrary{topaths,calc,automata,positioning}
\newtheorem*{theorem*}{Theorem}
\newtheorem{problem}{Problem}

\usepackage{listings}
\usepackage{xcolor} % for custom colors
\lstset{
    language=Python,
    basicstyle=\ttfamily\small,
    keywordstyle=\color{blue},
    commentstyle=\color{gray},
    stringstyle=\color{red},
    frame=single,
    breaklines=true,
    showstringspaces=false
}

\theoremstyle{definition}
\newtheorem*{example*}{Example}
\newtheorem*{solution*}{Solution}

\theoremstyle{remark}
\newtheorem*{remark*}{Remark}

\renewcommand{\implies}{\Longrightarrow\hspace{5mm}}
\renewcommand{\iff}{\hspace{5mm}\Longleftrightarrow\hspace{5mm}}

\setlength{\arraycolsep}{4pt}


%\fontsize{font size}{vertsize (usually 1.2x)}\selectfont

\begin{document}
\renewcommand{\familydefault}{\rmdefault}

\begin{titlepage}
    \begin{center}
    {\fontsize{40}{48}\selectfont \bfseries Problem Set 1} 
    \\\vspace{20pt}
    {\LARGE FINM 34000} \\
    \vspace{20pt}
    
    \vfill % Fills the vertical space above the image

        \includegraphics[width=0.75\textwidth]{Images/uchicago_logo.png}

        \vfill % Fills the vertical space below the image
        \textbf{Reece VanDeWeghe}

        Prepared for Sep 8, 2025
    \end{center}
\end{titlepage}
\pagestyle{fancy}
\fancyhf{}
\setlength{\headheight}{30pt}
\renewcommand{\headrulewidth}{0.4pt}
\renewcommand{\footrulewidth}{0.4pt}
\lhead{\large MIPC \\ 1st Round Proposal}
\rhead{\large Oct 7, 2025 \\ VanDeWeghe Hanna Moukabary}
\rfoot{\textbf{Page \thepage}}
\lfoot{}


%\tableofcontents
\pagebreak


\fontsize{12}{20}\selectfont{


% Purpose: Hook the judges immediately; make your dual mandate solution and fund vision crystal clear.

% Suggested structure:
% 	•	One-sentence purpose:
% “We propose a redesigned Borealis Wealth Fund that converts Finland’s forestry revenues into a self-sustaining biodiversity investment engine, balancing 5–6% annual financial returns with measurable biodiversity restoration by 2035.”
% 	•	3 pillars of your solution (e.g. “Biodiversity Integration, Portfolio Transformation, and Global Co-Leadership”).
% 	•	Key metrics: AUM (€10B), expected return, biodiversity targets (hectares restored, carbon sink recovery timeline).
% 	•	Snapshot of your portfolio allocation (e.g., pie chart).
% 	•	Signature innovation (e.g. “Natural Capital Performance Index” or “Biodiversity-Linked Bonds”).



% Summarize Finland’s biodiversity-finance challenge and define BWF’s reoriented role.

% Include:
% 	•	Finland’s forests → carbon sink turned source (2021).
% 	•	Dual mandate tension: biodiversity restoration vs. financial sustainability.
% 	•	Your strategic framing:
% “BWF must act as Finland’s Natural Capital Fund—converting ecological resilience into investable capital.”

% Deliverable: Define your investment philosophy (why biodiversity = value).
% E.g., “Biodiversity enhances asset durability and mitigates regulatory and stranded-asset risk—creating unpriced alpha.”

\section*{Mandate Reframing}

Finland’s forests, which cover over 75\% of the country’s land area, have long served as both an ecological foundation and an economic engine. However, in 2021 these forests shifted from being a net carbon sink to a net carbon source, driven by intensive harvesting and peatland degradation. This reversal jeopardizes Finland’s legally binding 2035 net-zero emissions target and threatens the economic stability of a sector responsible for nearly 4\% of national GDP and 18\% of exports. 

The Borealis Wealth Fund (BWF), with €10 billion in assets under management and a recurring inflow of approximately €1 billion per year from forestry-linked tax revenues, has been tasked with realigning the country’s natural capital strategy. Initially launched as a broad green investment vehicle, BWF must now address a dual mandate: to generate long-term, risk-adjusted returns while restoring the ecological resilience that underpins Finland’s bioeconomy. This represents not merely a portfolio adjustment, but a systemic reframing of how national wealth and nature interact.

We propose repositioning BWF as Finland’s \textbf{Natural Capital Sovereign Fund}—a platform that treats biodiversity as a core economic asset. Under this model, BWF channels forestry-derived revenues into investments that enhance ecosystem health, stabilize carbon sinks, and de-risk Finland’s export base from future EU biodiversity and land-use regulations. The Fund’s objective is to deliver sustainable financial returns (targeting 5–6\% per annum) while achieving measurable biodiversity outcomes, including the restoration of over 200,000 hectares of degraded forest and peatland by 2035.

This reframing rests on a clear investment philosophy:

\begin{quote}
\emph{Biodiversity enhances asset durability and long-term value creation by improving ecosystem productivity, regulatory resilience, and social license to operate. Protecting nature is not a constraint on returns—it is a catalyst for enduring value.}
\end{quote}

BWF’s redefined investment approach will be guided by three principles:
\begin{enumerate}[label=\textbf{\arabic*.}]
    \item \textbf{Integration:} Embed biodiversity risk and opportunity into every investment decision using frameworks such as the Taskforce on Nature-related Financial Disclosures (TNFD) and the EU Biodiversity Strategy for 2030.
    \item \textbf{Additionality:} Allocate at least 30\% of annual new capital inflows to projects that deliver measurable ecological benefits—such as habitat restoration, peatland recovery, and native species protection—that exceed baseline market activity.
    \item \textbf{Resilience:} Diversify across asset classes and geographies to reduce overexposure to Finland’s forestry cycle, ensuring long-term capital preservation and compliance with evolving EU nature restoration laws.
\end{enumerate}

Through this mandate, BWF evolves from a conventional green finance fund into a global model for biodiversity-aligned sovereign investment. By quantifying and monetizing the value of natural capital, BWF will enable Finland to transform ecological preservation into a sustainable source of fiscal strength and international leadership.

\section*{Diagnostic: Legacy Portfolio \& Gaps}

\subsection*{Where We Are}
Borealis Wealth Fund (BWF) manages \textbf{€10B} and has delivered an average \textbf{6.5\%} annual return since launch. The legacy portfolio is diversified across public markets, fixed income, and real assets, but ESG screening has not explicitly targeted biodiversity outcomes.

\begin{table}[h!]
\centering
\resizebox{0.9\linewidth}{!}{%
\begin{tabular}{lccc}
\toprule
\textbf{Asset Class} & \textbf{Share of AUM (\%)} & \textbf{Role Today} & \textbf{Biodiversity Link} \\
\midrule
Public Equities (ESG-screened) & \textasciitilde 40 & Growth \& liquidity & Indirect / weak \\
Green Bonds \& Thematic Credit & \textasciitilde 15 & Income \& duration & Mixed / label-driven \\
Green Infrastructure (wind/solar/transit) & \textasciitilde 20 & Inflation hedge, stable cash & Indirect (climate-first) \\
Forestry \& Land Holdings & \textasciitilde 15 & Real-asset beta & Direct, but unmanaged \\
Cash \& Liquidity & \textasciitilde 10 & Dry powder & None \\
\bottomrule
\end{tabular}%
}
\caption{Legacy portfolio overview by strategic sleeve (percent shares inferred from case totals).}
\end{table}

\noindent \textbf{Geography:} \textasciitilde 55\% Finland, 30\% EU ex-Finland, 10\% North America, 5\% EM.

\vspace{0.5em}
\subsection*{What Works}
\begin{itemize}[leftmargin=*, itemsep=2pt]
    \item \textbf{Solid core beta:} Low-cost public equity exposure and green infrastructure provide diversified return drivers and partial inflation protection.
    \item \textbf{Policy alignment (climate):} Green infrastructure and thematic credit align with climate transition risks and EU taxonomy to a degree.
    \item \textbf{Liquidity balance:} A meaningful liquid sleeve (equities, credit, cash) supports rebalancing and new program launches.
\end{itemize}
\subsection*{Binding Constraints }
\begin{itemize}[leftmargin=*, itemsep=2pt]
    \item \textbf{2035 Net-Zero Pathway:} Finland’s forests shifted from a carbon sink to a carbon source in 2021, compressing the national carbon budget and intensifying scrutiny on land-use and forestry practices.
    \item \textbf{EU Nature Regulation:} The EU Biodiversity Strategy and Nature Restoration Law increase compliance pressure, making traditional high-yield forestry less viable without demonstrable ecological outcomes.
    \item \textbf{Trade and Competitiveness:} International buyers, especially within the EU, increasingly demand traceable, biodiversity-positive supply chains. Failure to meet these standards risks market access and export competitiveness.
\end{itemize}

\subsection*{Identified Gaps}
\begin{enumerate}[leftmargin=*, itemsep=2pt]
    \item \textbf{ESG $\neq$ Biodiversity:} Current ESG screens do not capture habitat integrity, species abundance, deadwood density, or landscape connectivity. 
    \item \textbf{Logging Yields vs. Future EU Regulations:} EU biodiversity laws and restoration targets will likely constrain harvest volumes. The Fund’s return model must anticipate this by investing in higher-value wood products and wood-based innovation, enabling profitability even with lower yields.
    \item \textbf{Liquidity vs. Long-term Natural Assets:} Real assets such as forestland and green infrastructure generate durable impact but remain illiquid. Without sufficient liquidity, the Fund risks being unable to pivot or scale new biodiversity projects. A structured cash and green bond allocation can serve as \emph{impact working capital} to fund near-term transitions.
    \item \textbf{Domestic Exposure vs. Global Trade Partners:} While increasing Finland’s forestry investment supports domestic biodiversity goals, overconcentration exposes the Fund to local regulatory, ecological, and market shocks. International ESG equities and green infrastructure holdings mitigate this by sustaining diversified trade and financial linkages.
\end{enumerate}

\subsection*{Risk Map}
\begin{table}[h!]
\centering
\resizebox{0.9\linewidth}{!}{%
\begin{tabular}{p{3.3cm} p{6.6cm} p{3.3cm}}
\toprule
\textbf{Risk} & \textbf{Mechanism} & \textbf{Current Exposure} \\
\midrule
Regulatory / Policy & Tightening EU nature and restoration mandates impacting harvest volumes & High (forestry, land) \\
Physical / Ecological & Carbon sink reversal; biodiversity loss; soil and water stress & Medium–High (domestic forests) \\
Market / Reputation & Global buyers’ biodiversity standards; reputational risk from greenwashing & Medium (equities, bonds) \\
Liquidity / Funding & High share of illiquid real assets vs. short-term capital needs & Medium \\
Concentration / Home Bias & Exposure to Finland’s forestry and export cycles & High \\
\bottomrule
\end{tabular}
}
\caption{Principal risk channels and qualitative exposure assessment.}
\end{table}


\subsection*{Implications for Strategy }
\begin{itemize}[leftmargin=*, itemsep=2pt]
    \item Implement a strategy that focuses on net-zero goals while also addressing biodiversity through maintaining forested lands investments and reforesting new land.
    \item Establish a \emph{Value-over-Volume} logging model, reinvesting profits into R\&D and venture capital that increase wood product margins.
    \item Maintain 15\%–20\% of the portfolio in liquid green bonds and cash to ensure flexibility in meeting restoration and reforestation funding needs.
    \item Strengthen cross-border ESG equity exposure to sustain Finland’s trade relationships and hedge domestic forestry concentration risk.
    
\end{itemize}


% The intellectual core of your case.

% Framework idea: “The Biodiversity Investment Loop”
% 	1.	Screen: Identify biodiversity-linked risks/opportunities (e.g., TNFD, ENCORE).
% 	2.	Select: Invest in assets with measurable ecological value (using biodiversity KPIs).
% 	3.	Transform: Reinvest forestry revenues into high-impact restoration.
% 	4.	Report: Use a “Biodiversity-Adjusted Return (BAR)” framework to link financial and natural outcomes.

% You could visualize this as a 4-step loop or flow chart — visual clarity is huge in MIPC reports.

% Add your measurement proposal (e.g.):
% 	•	Biodiversity Impact Unit (BIU): € invested per hectare restored or per species recovered.
% 	•	Portfolio Biodiversity Intensity (PBI): weighted biodiversity exposure per asset class.

% Make it concrete and financial — this is where top teams shine.

% a. New Target Allocation (Table + Pie Chart)

% Total: €10B
% Target Return: ~6%
% Target Impact: restore 200,000+ ha, reduce forest CO₂ emissions by 30% by 2035.

% b. Justify Each Block
% 	•	Link to case trade-offs (logging yield, concentration risk, regulatory uncertainty, short-term cashflow vs. long-term resilience).
% 	•	Mention diversification across:
% 	•	Geography (Finland + global impact)
% 	•	Sector (forestry, carbon, tech, ecotourism)
% 	•	Instrument type (equity, debt, real assets)

% c. Scenario Analysis (Optional but powerful)

% Brief 2x2 table: Baseline / Accelerated Biodiversity / Regulatory Shock / High-Yield Forestry — show resilience of your portfolio.

\section{Portfolio Allocation}
\subsection{ESG focused equities}
\par We propose allocating 50\% of the portfolio to ESG-focused global equities. This represents an increase compared to the legacy allocation and serves to balance lower-yield, sustainability-driven investments in forestry and land. The structure and philosophy of this equity allocation will mirror that of the Norwegian Government Pension Fund Global (GPFG)—commonly known as the Norwegian Oil Fund—due to its strong ESG orientation, diversified exposure, and proven track record of long-term performance.
\par The GPFG’s historical returns provide a benchmark for our expectations. Over the past 15 years, the fund has averaged between 10–13\% annual returns, with a recent five-year average of 13.83\%, as shown below:\\
\bigskip
\begin{center}
    

\begin{tabular}{ |p{3cm}||p{3cm}|p{3cm}|p{3cm}|  }
 \hline
 \multicolumn{4}{|c|}{Norwegian Fund Returns table} \\
 \hline
  & Since 2019 & Since 2014 & Since 2009\\
 \hline
 Returns   & 13.83\%    &10.31\%&  12.28\%\\
 
 \hline
\end{tabular}
\end{center}
\bigskip

\par We therefore estimate an annualized return of approximately 12\% for this portion of the portfolio, justified by the fund’s global diversification and exposure to ESG sectors poised for structural growth, such as renewable energy, sustainable technology, and low-carbon infrastructure.


\subsection{Forestry and Land investments}
As stated above, the core mission of this strategy is to achieve carbon neutrality while addressing biodiversity loss that is incurred through deforestation. The solution to this problem is two-fold. First, increase the investment in forested lands and lands that can be reforested to 20\%. Second, slow deforestation by shifting focus from quantity to quality (discussed in the next section). Although it is somewhat difficult to measure the returns on this type of land, over the past eight years the value of this land has increased by 54\% in Finland (roughly 5.5\% year over year). For this part of the portfolio, we estimate an annual return of 6\%. This is based on the expected increase in the value of the land itself and the annual return of forestland. This estimate is somewhat conservative, based on the fact that we do not expect to see the same return as private forest investments (do to our conservation efforts).

\subsection{Green Infrastructure and Venture Capitalism}
The second part of our solution is to invest in companies that will make the forestry industry more efficient and have climate focused infrastructure. 15\% of the portfolio will be invested in green infrastructure and forestry-related venture capital. These investments will target companies and projects that increase the value per unit of wood, improve processing efficiency, and expand the use of sustainable materials across industries.
Benchmark data from the Norwegian GPFG’s unlisted infrastructure portfolio shows 2.8\% annualized returns over 4.5 years. However, given the higher uncertainty and illiquidity associated with venture-stage investments, we adopt a 2\% expected return for this category. These investments, though relatively low-yielding in the short term, are critical to achieving long-term sustainability and technological leadership within Finland’s forestry sector.


\subsection{Green Bonds}
10\% of the portfolio will be allocated to green bonds, including both sovereign and corporate issues. These instruments provide stable, low-volatility returns while reinforcing the fund’s ESG commitments. Historically, global green bond yields have ranged between 2–2.5\%, though certain emerging-market and municipal green bonds (e.g., India’s 2024 issuance) offer yields of up to 7\%.
Given increasing issuance volume and yield diversification in the green bond market, we estimate a 4\% average annual return for this asset class. Beyond return considerations, this allocation enhances liquidity and serves as a stabilizing anchor within the broader portfolio.

\subsection{Cash}
The remaining 5\% of the portfolio will be held in cash and short-term liquidity instruments. This position serves multiple purposes: 
\begin{itemize}
    \item To provide flexibility for opportunistic investments, particularly in forestry and infrastructure;
    \item To act as a risk buffer in volatile markets; and 
    \item To cover operational and rebalancing needs without forced asset sales.
\end{itemize}
Given current short-term interest rate levels, we estimate a 2\% return on cash holdings.

\subsection{Total Portfolio Expected Returns}
The overall portfolio is designed to balance long-term growth with sustainability and resilience. Using the weighted average of expected returns from each asset class, the portfolio’s total expected annual return is approximately 8.4\%. This estimate aligns with the performance range of major sovereign and ESG-focused institutional funds while embedding stronger environmental impact metrics. 

\bigskip
\begin{center}
\begin{tabular}{ |p{4cm}||p{2cm}|p{4cm}|  }
 \hline
 \multicolumn{3}{|c|}{Expected Portfolio Performance Summary} \\
 \hline
 Asset Class & Allocation & Est. Annual Return \\
 \hline
 ESG Equities & 50\% & 12\% \\
 Forestry \& Land & 20\% & 6\% \\
 Green Infrastructure \& VC & 15\% & 2\% \\
 Green Bonds & 10\% & 4\% \\
 Cash & 5\% & 2\% \\
 \hline
 \textbf{Weighted Average Return} & 100\% & \textbf{8.4\%} \\
 \hline
\end{tabular}
\end{center}

\bigskip
This portfolio positions BWF as a leader in nature-positive finance—demonstrating that environmental restoration and economic performance can be mutually reinforcing goals.

\bibliography{Layout/Sources}
\bibitem[1]{https://www.nbim.no/en/investments/returns/}
\bibitem[2]{https://sijoitusmetsat.fi/en/press-release-forest-property-market-prices-increased-modestly-in-2024/}
\bibitem[3]{Green Bonds: New Label, Same Projects, Pauline Lam, Jeffrey Wurgler}

% Show feasibility and timeline.

% Example:
% 	•	2025–2026: Transition legacy assets, launch pilot biodiversity fund (€500M).
% 	•	2027–2030: Scale biodiversity-linked bonds; co-invest with EU Life Programme.
% 	•	2031–2035: Reach biodiversity-positive forest balance; publish first Natural Capital Report.



% Tie it together.

% 	•	Key risks: regulatory (EU law), liquidity, data uncertainty.
% 	•	Mitigations: blended finance, public-private co-investment, adaptive management.
% 	•	Closing statement:
% “By pricing biodiversity into Finland’s sovereign wealth, BWF can turn ecological resilience into a global competitive advantage.”

}
\par The BWF Asset Management Strategy was designed to deliver long-term returns while supporting Finland’s carbon-neutral ambitions. However, the nation’s timber industry—central to its economy—often conflicts with these goals, and ongoing deforestation has accelerated biodiversity loss. Addressing these challenges requires a balanced approach that integrates environmental restoration with economic sustainability.
\par The revised strategy seeks to achieve carbon neutrality through biodiversity, not despite it. BWF will invest in native forest restoration and reforestation projects that enhance long-term carbon capture and ecosystem health, generating returns through land value appreciation and future sustainable timber revenues.
\par To balance environmental and economic priorities, the fund will back companies and innovations that raise the value per unit of wood, enabling reduced harvest volumes without sacrificing profitability. While increasing domestic exposure to Finland’s forests, BWF will maintain diversification through international equities and green infrastructure to manage risk.
\par The fund will also scale biodiversity-focused projects with lower immediate returns, offsetting them through strong performance in ESG-aligned equities, and will promote sustainable trade partnerships, advocating EU preferences for conservation-certified Finnish timber.
\par The \$10 billion portfolio allocates 50\% to global ESG equities, 20\% to forestry and land holdings, 15\% to green infrastructure, 10\% to green bonds, and 5\% to cash.
This balanced strategy positions BWF as a global model for nature-positive finance, demonstrating that economic prosperity, carbon neutrality, and biodiversity enhancement can reinforce—rather than compete with—one another.
%\printbibliography

\end{document}