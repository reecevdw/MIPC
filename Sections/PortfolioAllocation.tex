% Make it concrete and financial — this is where top teams shine.

% a. New Target Allocation (Table + Pie Chart)

% Total: €10B
% Target Return: ~6%
% Target Impact: restore 200,000+ ha, reduce forest CO₂ emissions by 30% by 2035.

% b. Justify Each Block
% 	•	Link to case trade-offs (logging yield, concentration risk, regulatory uncertainty, short-term cashflow vs. long-term resilience).
% 	•	Mention diversification across:
% 	•	Geography (Finland + global impact)
% 	•	Sector (forestry, carbon, tech, ecotourism)
% 	•	Instrument type (equity, debt, real assets)

% c. Scenario Analysis (Optional but powerful)

% Brief 2x2 table: Baseline / Accelerated Biodiversity / Regulatory Shock / High-Yield Forestry — show resilience of your portfolio.

\section{Portfolio Allocation}
\subsection{ESG focused equities}
\par We propose allocating 50\% of the portfolio to ESG-focused global equities. This represents an increase compared to the legacy allocation and serves to balance lower-yield, sustainability-driven investments in forestry and land. The structure and philosophy of this equity allocation will mirror that of the Norwegian Government Pension Fund Global (GPFG)—commonly known as the Norwegian Oil Fund—due to its strong ESG orientation, diversified exposure, and proven track record of long-term performance.
\par The GPFG’s historical returns provide a benchmark for our expectations. Over the past 15 years, the fund has averaged between 10–13\% annual returns, with a recent five-year average of 13.83\%, as shown below:\\
\bigskip
\begin{center}
    

\begin{tabular}{ |p{3cm}||p{3cm}|p{3cm}|p{3cm}|  }
 \hline
 \multicolumn{4}{|c|}{Norwegian Fund Returns table} \\
 \hline
  & Since 2019 & Since 2014 & Since 2009\\
 \hline
 Returns   & 13.83\%    &10.31\%&  12.28\%\\
 
 \hline
\end{tabular}
\end{center}
\bigskip

\par We therefore estimate an annualized return of approximately 12\% for this portion of the portfolio, justified by the fund’s global diversification and exposure to ESG sectors poised for structural growth, such as renewable energy, sustainable technology, and low-carbon infrastructure.


\subsection{Forestry and Land investments}
As stated above, the core mission of this strategy is to achieve carbon neutrality while addressing biodiversity loss that is incurred through deforestation. The solution to this problem is two-fold. First, increase the investment in forested lands and lands that can be reforested to 20\%. Second, slow deforestation by shifting focus from quantity to quality (discussed in the next section). Although it is somewhat difficult to measure the returns on this type of land, over the past eight years the value of this land has increased by 54\% in Finland (roughly 5.5\% year over year). For this part of the portfolio, we estimate an annual return of 6\%. This is based on the expected increase in the value of the land itself and the annual return of forestland. This estimate is somewhat conservative, based on the fact that we do not expect to see the same return as private forest investments (do to our conservation efforts).

\subsection{Green Infrastructure and Venture Capitalism}
The second part of our solution is to invest in companies that will make the forestry industry more efficient and have climate focused infrastructure. 15\% of the portfolio will be invested in green infrastructure and forestry-related venture capital. These investments will target companies and projects that increase the value per unit of wood, improve processing efficiency, and expand the use of sustainable materials across industries.
Benchmark data from the Norwegian GPFG’s unlisted infrastructure portfolio shows 2.8\% annualized returns over 4.5 years. However, given the higher uncertainty and illiquidity associated with venture-stage investments, we adopt a 2\% expected return for this category. These investments, though relatively low-yielding in the short term, are critical to achieving long-term sustainability and technological leadership within Finland’s forestry sector.


\subsection{Green Bonds}
10\% of the portfolio will be allocated to green bonds, including both sovereign and corporate issues. These instruments provide stable, low-volatility returns while reinforcing the fund’s ESG commitments. Historically, global green bond yields have ranged between 2–2.5\%, though certain emerging-market and municipal green bonds (e.g., India’s 2024 issuance) offer yields of up to 7\%.
Given increasing issuance volume and yield diversification in the green bond market, we estimate a 4\% average annual return for this asset class. Beyond return considerations, this allocation enhances liquidity and serves as a stabilizing anchor within the broader portfolio.

\subsection{Cash}
The remaining 5\% of the portfolio will be held in cash and short-term liquidity instruments. This position serves multiple purposes: 
\begin{itemize}
    \item To provide flexibility for opportunistic investments, particularly in forestry and infrastructure;
    \item To act as a risk buffer in volatile markets; and 
    \item To cover operational and rebalancing needs without forced asset sales.
\end{itemize}
Given current short-term interest rate levels, we estimate a 2\% return on cash holdings.

\subsection{Total Portfolio Expected Returns}
The overall portfolio is designed to balance long-term growth with sustainability and resilience. Using the weighted average of expected returns from each asset class, the portfolio’s total expected annual return is approximately 8.4\%. This estimate aligns with the performance range of major sovereign and ESG-focused institutional funds while embedding stronger environmental impact metrics. 

\bigskip
\begin{center}
\begin{tabular}{ |p{4cm}||p{2cm}|p{4cm}|  }
 \hline
 \multicolumn{3}{|c|}{Expected Portfolio Performance Summary} \\
 \hline
 Asset Class & Allocation & Est. Annual Return \\
 \hline
 ESG Equities & 50\% & 12\% \\
 Forestry \& Land & 20\% & 6\% \\
 Green Infrastructure \& VC & 15\% & 2\% \\
 Green Bonds & 10\% & 4\% \\
 Cash & 5\% & 2\% \\
 \hline
 \textbf{Weighted Average Return} & 100\% & \textbf{8.4\%} \\
 \hline
\end{tabular}
\end{center}

\bigskip
This portfolio positions BWF as a leader in nature-positive finance—demonstrating that environmental restoration and economic performance can be mutually reinforcing goals.

% \bibliography{Layout/Sources}
% \bibitem[1]{https://www.nbim.no/en/investments/returns/}
% \bibitem[2]{https://sijoitusmetsat.fi/en/press-release-forest-property-market-prices-increased-modestly-in-2024/}
% \bibitem[3]{Green Bonds: New Label, Same Projects, Pauline Lam, Jeffrey Wurgler}